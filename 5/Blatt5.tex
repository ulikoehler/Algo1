\documentclass[a4paper,10pt,oneside,leqno]{scrartcl}
\usepackage[utf8]{inputenc}
\usepackage{listingsutf8}
\usepackage[pdftex]{graphicx}
\usepackage[ngerman]{babel}
\usepackage{url}
\usepackage{hyperref}
\usepackage{amsfonts}
\usepackage{amssymb}
\usepackage{amsmath}
\usepackage{tikz}
\usepackage{listings}
\usepackage{textcomp}
\usepackage{amsmath}
%\usepackage{lipsum}%%%%%%%%%%%%%%%%%%%%%%%%%%%%%%%%%%%%%%%%%%%%%%%%%%%%%%%%%%%%%%%%%%%%%%%%%%%%%%%%%%%%
\usepackage{mathtools}
\usepackage{amsfonts}
\usepackage{amssymb}
\usepackage{amsmath}
\usepackage{booktabs}
%usepackage[utf8x]{inputenc}
\usepackage[T1]{fontenc}
\usepackage{lmodern} %Latin modern = enhanced CM font
\usepackage{xspace} %Space enhancements
%\usepackage{algorithm}%%%%%%%%%%%%%%%%%%%%%%%%%%%%%%%%%%%%%%%%%%%%%%%%%%%%%%%%%%%%%%%%%%%%%%%%%%%%%%%%%%
\usepackage{algpseudocode}
\renewcommand{\thefootnote}{\fnsymbol{footnote}}
\definecolor{mygreen}{rgb}{0,0.6,0}
\definecolor{mygray}{rgb}{0.5,0.5,0.5}
\definecolor{mymauve}{rgb}{0.58,0,0.82}
\lstset{ %
  backgroundcolor=\color{white},   % choose the background color; you must add \usepackage{color} or \usepackage{xcolor}
  basicstyle=\footnotesize,        % the size of the fonts that are used for the code
  breakatwhitespace=false,         % sets if automatic breaks should only happen at whitespace
  breaklines=true,                 % sets automatic line breaking
  captionpos=b,                    % sets the caption-position to bottom
  commentstyle=\color{mygreen},    % comment style
  deletekeywords={...},            % if you want to delete keywords from the given language
  escapeinside={\%*}{*)},          % if you want to add LaTeX within your code
  extendedchars=true,              % lets you use non-ASCII characters; for 8-bits encodings only, does not work with UTF-8
  frame=single,                    % adds a frame around the code
  keepspaces=true,                 % keeps spaces in text, useful for keeping indentation of code (possibly needs columns=flexible)
  keywordstyle=\color{blue},       % keyword style
  language=Octave,                 % the language of the code
  morekeywords={*,...},            % if you want to add more keywords to the set
  numbers=left,                    % where to put the line-numbers; possible values are (none, left, right)
  numbersep=5pt,                   % how far the line-numbers are from the code
  numberstyle=\tiny\color{mygray}, % the style that is used for the line-numbers
  rulecolor=\color{black},         % if not set, the frame-color may be changed on line-breaks within not-black text (e.g. comments (green here))
  showspaces=false,                % show spaces everywhere adding particular underscores; it overrides 'showstringspaces'
  showstringspaces=false,          % underline spaces within strings only
  showtabs=false,                  % show tabs within strings adding particular underscores
  stepnumber=2,                    % the step between two line-numbers. If it's 1, each line will be numbered
  stringstyle=\color{mymauve},     % string literal style
  tabsize=2,                       % sets default tabsize to 2 spaces
  title=\lstname                   % show the filename of files included with \lstinputlisting; also try caption instead of title
}

%opening
\title{Übungsblatt 4}
\author{Uli Köhler (10580373), Tobias Harrer (10575835)}
\begin{document}

\maketitle

\section*{Aufgabe 1}
Unterschied zwischen der herkömmlichen $border$-Tabelle und der neuen $border'$-Tabelle: Die neue $border'$-Tabelle berechnet auch den
längsten echten Rand, bis auf den Unterschied, dass das letzte Zeichen unterschiedlich ist. Das heißt, wir können auch hier $border[j]$
bestimmen, in dem wir prüfen, ob der eig. Rand von $s_0...s_{j-1}$ mit $s_j$ erweiterbar ist. Allerdings muss die Bedingung in der while-
Schleife geändert werden. Diese bricht bisher ab, wenn $i<0$ wird, d.h. $i=-1=border[0]$ und kein eigentlicher Rand existiert, oder wenn
$s[i]$ und $s[j-1]$ übereinstimmen. Im Fall von $border'$ jedoch soll die Bedingung gelten, dass sich der letzte Buchstabe unterscheidet,
also muss die while-Schleife abbrechen, wenn $i<0$ oder wenn $s[i]$ und $s[j-1]$ \textit{nicht} übereinstimmen. Dies ist zulässig, da wir
ein binäres Alphabet verwenden, und ein Nichtübereinstimmen der Buchstaben bedeudet, dass der jew. andere Buchstabe bei $s_j$ steht, was die
geforderte Bedingung darstellt.
\begin{algorithmic}
\Function{compute border'}{int[] $border'$, int $m$, char[] $s$}
\State $border'[0] := -1$;
\State $border'[1] := 0$;
\State int $i := border'[1]$;
\For{(int $j := 2$; $j\leq m$; $j ++$)}
  \While{($(i\geq 0) \&\& (s[i] = s[j-1])$)}
    \State $i := border'[i]$;
  \EndWhile
  \State $i++$;
  \State $border'[j] := i;$
\EndFor
\EndFunction
\end{algorithmic}
$border'$ für Blatt 4, Aufgabe 2:\newline
\begin{tabular}{l|c|r}
\textbf{j} & \textbf{Präfix} & \textbf{$border'[j]$}\\\hline
0 &\$ & -1\\
1 &a\$ & 0\\
2 &ab\$ & 1\\
3 &aba\$ & 0\\
4 &abaa\$ &  2\\
5 &abaab\$  & 1\\
6 &abaaba\$  & 0\\
7 &abaabab\$ & 4\\
8 &abaababa\$ & 0\\
9 &abaababab\$ & 4\\
10&abaabababa\$ & 0\\
11&abaabababab\$ & 4\\
12&abaababababa\$ & 0\\
13&abaababababaa\$ & 2\\
14&abaababababaab\$ & 1\\
15&abaababababaabb\$ & 5\\
\end{tabular}\newline\newline
\textit{Laufzeit}: vgl. 2.2.7 im Skript: maximal $m-1$ erfolgreiche Vergleiche, da jedes Mal $j\in [2 : m]$ um
1 erhöht und nie erniedrigt wird. Für erfolglose Vergleiche: $i$ ist anfangs gleich 0, wird ($m-1$)
Mal um 1 erhöht, da die for-Schleife ($m-1$) Mal durchlaufen wird. $i$ kann maximal $(m-1) + 1 = m$ Mal erniedrigt werden, da immer $i\geq -1$ gilt.
Also ist die Anzahl der Vergleiche durch $2m-1\in \mathcal{O}(m)$ beschränkt.\newline\newline
\textit{Korrektheit}: Beim Berechnen der $border'$-Tabelle für $abaababababaabb$ fiel auf, dass der obige Algorithmus nicht korrekt ist. Jedoch
muss er in etwa diese Struktur besitzen (while Schleife in der for-Schleife), um $\mathcal{O}(m)$ zu erfüllen. Das Problem liegt darin, dass der
eig. Rand von $s_0...s{j-1}$ nicht einfach erweitert werden kann.

\section*{Aufgabe 2}
\begin{algorithmic}
\Function{bool KMP}{char[] $t$, int $n$, char[] $s$, int $m$}
\State int $border'[m+1];$
\State COMPUTE BORDER'(int $border'[]$, int $m$, char $s[]$);
\State int $i := 0$,$j:=0;$
\While{($i\leq n-m$)}
  \While{($t[i+j] = s[j]$)}
    \State $j++;$
    \If {($j = m$)} \State \Return TRUE;\EndIf
  \EndWhile
  \State $i := i + (j-border'[j])$;
  \State $j :=$ max$(0, border'[j]);$
\EndWhile
\State \Return FALSE;
\EndFunction
\end{algorithmic}
Der KMP-Algorithmus kann aus dem Skript übernommen werden, bis auf die Berechnung der $border'$-Tabelle.\newline
\textit{Laufzeit:} (vgl. Skript) maximal $n-m+1$ erfolglose Vergleiche, da nach jedem erfolglosen Vergleich $i\in [0 : n- m]$ erhöht und
im Verlauf des Algorithmus nie erniedrigt wird. Der Wert von $i+j$ wird nie erniedrigt. Seien $i$ und $j$ die Werte vor und $i'$ und $j'$
die Werte nach einem erfolglosen Vergleich, mit $i' + j' = (i + j- border'[j]) + (max(0, border'[j]))$.\newline
Fall 1 ($border'[j] \geq 0$): Ist $border'[j] \geq 0$, dann gilt offensichtlich $i'+j' = i + j$.\newline
Fall 2 ($border'[j] < 0$): Ist $border[j] = -1$, dann muss $j = 0 = j'$ sein. Dann gilt
aber offensichtlich $i'+j' = i' + 0 = (i + 0 -(-1)) + 0 = i + 1 = i + j + 1$.\newline
Also wird i + j nach einem erfolglosen Vergleich nicht kleiner, nach einem erfolgreichen Vergleich wird $i + j$ um 1 erhöht.
Somit werden insgesamt maximal $2n- m + 1$ Vergleiche ausgeführt.\newline
\textit{Korrektheit:}
\begin{itemize}
  \item $s$ ist in $t$ enthalten und Suffix von $t$:\newline Die Bedingung der zweiten while-Schleife wird für alle $j\in [0:m-1]$ erfüllt sein,
  der Fall $j=m$ wird eintreten und es wird TRUE zurückgegeben.
  \item $s$ ist in $t$ enthalten und nicht Suffix von $t$: Es gibt ein $j\in [0:m-1]$, an dem ein Mismatch auftreten wird, dann wird $i$ um
  $j-border'[j]$ weitergeschoben und $j$ auf max($0,border'[j]$) gesetzt. $border'[j]$ entspricht der Anzahl
\end{itemize}

\section*{Aufgabe 3}

\begin{tabular}{l|l|p{8cm}|c}
\textbf{i} & \textbf{Präfix inkl. Mismatch} & \textbf{Schritte} & \textbf{$ShiftTabelle[i]$}\\\hline
0 &\$ & Nicht anwendbar\\
1 &b\$ & Siehe (1) & 23\\
2 &ba\$ & Siehe (1) & 23\\
3 &bab\$ & Siehe (1) & 23\\
4 &babb\$ & Siehe (1) & 23\\
5 &babba\$ & Siehe (1) & 23\\
6 &babbaa\$ & Siehe (1) & 23\\
7 &babbaab\$ & Siehe (1) & 23\\
8 &babbaaba\$ & Siehe (1) & 23\\
9 &babbaabab\$ & Siehe (1) & 23\\
10&babbaababb\$ & Siehe (1) & 23\\
11&babbaababba\$ & Siehe (1) & 23\\
12&babbaababbab\$ & Siehe (1) & 23\\
13&babbaababbabb\$ & Siehe (1) & 23\\
14&babbaababbabba\$ & Siehe (1) & 23\\
15&babbaababbabbaa\$ & Siehe (1) & 23\\
16&babbaababbabbaab\$ & Siehe (1) & 23\\
17&babbaababbabbaaba\$ & Siehe (1)& 23\\
18&babbaababbabbaabab\$ & Siehe (1) & 23\\
19&babbaababbabbaababb\$ & Siehe (1) & 23\\
20&babbaababbabbaababba\$ & Siehe (1) & 23\\
21&babbaababbabbaababbaa\$ & Siehe (1) & 23\\
22&babbaababbabbaababbaab\$ & Das Alphabet ist [ab], daher wird nach dem Substring abab gesucht, der 2 positionen vorher vorhanden ist. & 2\\
23&babbaababbabbaababbaaba\$ & Das Alphabet ist [ab], daher wird nach dem Substring bba gesucht --> 5 pos vorher & 5\\
24&babbaababbabbaababbaabab\$ & Das Alphabet ist [ab], daher wird nach dem Substring aa gesucht, das 4 Pos. zuvor vorkommt & 4\\
25&babbaababbabbaababbaababa\$ & Kein Good Suffix vorhanden, & 1
\end{tabular}

\begin{itemize}
 \item (1): Pattern kann komplett verschoben werden, da der zu suchende Substring (der, s.u., eindeutig bestimmt werden kann), im Pattern nicht mehr vorkommt.
 \item Da das Alphabet nur die Größe 2 hat, kann der zu suchende Substring immer eindeutig bestimmt werden.
\end{itemize}
\end{document}