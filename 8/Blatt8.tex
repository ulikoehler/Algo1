\documentclass[a4paper,10pt,oneside,leqno]{scrartcl}
\usepackage[utf8]{inputenc}
\usepackage{listingsutf8}
\usepackage[pdftex]{graphicx}
\usepackage[ngerman]{babel}
\usepackage{url}
\usepackage{hyperref}
\usepackage{amsfonts}
\usepackage{amssymb}
\usepackage{amsmath}
\usepackage{tikz}
\usepackage{listings}
\usepackage{textcomp}
\usepackage{amsmath}
%\usepackage{lipsum}%%%%%%%%%%%%%%%%%%%%%%%%%%%%%%%%%%%%%%%%%%%%%%%%%%%%%%%%%%%%%%%%%%%%%%%%%%%%%%%%%%%%
\usepackage{mathtools}
\usepackage{amsfonts}
\usepackage{amssymb}
\usepackage{amsmath}
\usepackage{booktabs}
%usepackage[utf8x]{inputenc}
\usepackage[T1]{fontenc}
\usepackage{lmodern} %Latin modern = enhanced CM font
\usepackage{xspace} %Space enhancements
%\usepackage{algorithmic}%%%%%%%%%%%%%%%%%%%%%%%%%%%%%%%%%%%%%%%%%%%%%%%%%%%%%%%%%%%%%%%%%%%%%%%%%%%%%%%%%%
\usepackage{algpseudocode}
\renewcommand{\thefootnote}{\fnsymbol{footnote}}
\definecolor{mygreen}{rgb}{0,0.6,0}
\definecolor{mygray}{rgb}{0.5,0.5,0.5}
\definecolor{mymauve}{rgb}{0.58,0,0.82}
\lstset{ %
  backgroundcolor=\color{white},   % choose the background color; you must add \usepackage{color} or \usepackage{xcolor}
  basicstyle=\footnotesize,        % the size of the fonts that are used for the code
  breakatwhitespace=false,         % sets if automatic breaks should only happen at whitespace
  breaklines=true,                 % sets automatic line breaking
  captionpos=b,                    % sets the caption-position to bottom
  commentstyle=\color{mygreen},    % comment style
  deletekeywords={...},            % if you want to delete keywords from the given language
  escapeinside={\%*}{*)},          % if you want to add LaTeX within your code
  extendedchars=true,              % lets you use non-ASCII characters; for 8-bits encodings only, does not work with UTF-8
  frame=single,                    % adds a frame around the code
  keepspaces=true,                 % keeps spaces in text, useful for keeping indentation of code (possibly needs columns=flexible)
  keywordstyle=\color{blue},       % keyword style
  language=Octave,                 % the language of the code
  morekeywords={*,...},            % if you want to add more keywords to the set
  numbers=left,                    % where to put the line-numbers; possible values are (none, left, right)
  numbersep=5pt,                   % how far the line-numbers are from the code
  numberstyle=\tiny\color{mygray}, % the style that is used for the line-numbers
  rulecolor=\color{black},         % if not set, the frame-color may be changed on line-breaks within not-black text (e.g. comments (green here))
  showspaces=false,                % show spaces everywhere adding particular underscores; it overrides 'showstringspaces'
  showstringspaces=false,          % underline spaces within strings only
  showtabs=false,                  % show tabs within strings adding particular underscores
  stepnumber=2,                    % the step between two line-numbers. If it's 1, each line will be numbered
  stringstyle=\color{mymauve},     % string literal style
  tabsize=2,                       % sets default tabsize to 2 spaces
  title=\lstname                   % show the filename of files included with \lstinputlisting; also try caption instead of title
}

%opening
\title{Übungsblatt 8}
\author{Uli Köhler (10580373), Tobias Harrer (10575835)}
\begin{document}

\maketitle
\section*{Aufgabe 1}
%U%
\section*{Aufgabe 2}
\begin{algorithmic}
 \Function{int $find$}{String $s$, String $t$}
  \State SuffixTree $st$ = $buildSuffixTree$($t$); //Erzeugt einen Suffix-Baum
  \State Node $node$ = $getRoot$($t$);
  \State Edge $edge$ = $firstEdge$($node$);
  \State int $p$ = 0; //Die Position, an der $s$ verglichen wird
  \While{($node$ $\neq$ null AND $edge$ $\neq$ null)}
    \State $edge$ = $firstEdge$($node$);
    \While{($edge$ $\neq$ null)}
      \State (int, int) $label$ = $getEdgeLabel$($edge$); //int-Tupel
      \If{($s_{p..p+(label[1]-label[0]+1)}$ == $t_{label[0]..label[1]}$)} *
	\State $p$ := $p+(label[1]-label[0]+1)$; **
	\State break; //gehe aus while-schleife heraus
      \EndIf
      \State $edge$ = $nextEdge$($node$, $edge$);//Betrachte nächstes Label
    \EndWhile
    \If{($p\geq |s|$)}
    \State break;
    \Else
    \State $node$ = $getChild$($edge$); ***
    \EndIf
  \EndWhile
  \If{($p \geq length(s)$)}
    \State return $tiefenSuche(node)$;****
  \Else
    \State return 0;
  \EndIf
 \EndFunction
\end{algorithmic}
*//Falls $s$ und $t$ an geg. Pos. übereinstimmen\newline
**//erhöhe Pos. an der $s$ mit Label verglichen wird\newline
***//Falls passende Kante gefunden, gehe eine Ebene tiefer\newline
****//Zählt die Blätter des Teilbaums mit Wurzel $node$ == Anzahl der Vorkommen von $s$ in $t$ (vgl. Skript 3.4)\newline

\begin{itemize}
 \item Laufzeit: Zuerst wird der Suffix Tree berechnet (SuffixTree $st$ = $buildSuffixTree$($t$)), was nach Theorem 3.16/3.17 in linearer
 Zeit geschehen kann. Dann kommen im Wesentlichen zwei ineinandergeschachtelte while-Schleifen vor, die im schlechtesten Fall alle
 Kanten eines Knotens besuchen. Der Suffix Tree von $t$ hat maximal $2\cdot |t|$ Knoten, also werden echt weinger als $2\cdot |t|-1$ Kanten besucht,
 weil pro Ebene maximal ein Knoten besucht wird. Bei jedem Kantenbesuch werden Zeichenvergleiche zwischen $s$ und $t$ ausgeführt, aber maximal
 $|s|$ Vergleiche, da $|s| \leq |t|$. Also beinhalten die beiden while-Schleifen eine Komplexität, die in $\mathcal{O}(|s| \cdot |t|)$ liegt.
 Die anschließende Tiefensuche durchläuft max. $\mathcal{O}(|t|)$ Kanten. Also insgesamt $\mathcal{O}(|s| \cdot |t|)$, was aber nicht die effizienteste
 Methode ist. 
 
 \item Korrektheit: Der Algorithmus betrachtet von der Wurzel ausgehend jede Kante eines einzigen Knotens pro Ebene. Stimmt der zum Label einer Kante
 korrespondierende Substring aus $t$ mit $s$ vollständig überein, so wird die Anzahl der Blätter des von diesem Knoten beschriebenen Teilbaums in der
 Tiefensuche festgestellt. Stimmt $s$ nur zu einer Position $p$ < $|s|$ mit dem dem Label korrespondierende Substring aus $t$ überein, so wird der Knoten
 aufgesucht, zu dem diese Kante führt, und das Label gesucht, das $s$ weiter vervollständigt. Ist das der Fall, wird wie oben die Tiefensuche durchgeführt.
 Ist $s$ kein Substring von $t$, dann existiert kein Pfad von der Wurzel zu einem Knoten bzw ``in eine Kante'', und die Position $p$ wird immer echt
 kleiner als $|s|$ bleiben.
\end{itemize}

\section*{Aufgabe 3}
%T

\end{document}
