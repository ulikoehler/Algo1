\documentclass[a4paper,10pt,oneside,leqno]{scrartcl}
\usepackage[utf8]{inputenc}
\usepackage{listingsutf8}
\usepackage[pdftex]{graphicx}
\usepackage[ngerman]{babel}
\usepackage{url}
\usepackage{hyperref}
\usepackage{amsfonts}
\usepackage{amssymb}
\usepackage{amsmath}
\usepackage{tikz}
\usepackage{listings}
\usepackage{textcomp}
\usepackage{amsmath}
%\usepackage{lipsum}%%%%%%%%%%%%%%%%%%%%%%%%%%%%%%%%%%%%%%%%%%%%%%%%%%%%%%%%%%%%%%%%%%%%%%%%%%%%%%%%%%%%
\usepackage{mathtools}
\usepackage{amsfonts}
\usepackage{amssymb}
\usepackage{amsmath}
\usepackage{booktabs}
%usepackage[utf8x]{inputenc}
\usepackage[T1]{fontenc}
\usepackage{lmodern} %Latin modern = enhanced CM font
\usepackage{xspace} %Space enhancements
%\usepackage{algorithmic}%%%%%%%%%%%%%%%%%%%%%%%%%%%%%%%%%%%%%%%%%%%%%%%%%%%%%%%%%%%%%%%%%%%%%%%%%%%%%%%%%%
\usepackage{algpseudocode}
\renewcommand{\thefootnote}{\fnsymbol{footnote}}
\definecolor{mygreen}{rgb}{0,0.6,0}
\definecolor{mygray}{rgb}{0.5,0.5,0.5}
\definecolor{mymauve}{rgb}{0.58,0,0.82}
\lstset{ %
  backgroundcolor=\color{white},   % choose the background color; you must add \usepackage{color} or \usepackage{xcolor}
  basicstyle=\footnotesize,        % the size of the fonts that are used for the code
  breakatwhitespace=false,         % sets if automatic breaks should only happen at whitespace
  breaklines=true,                 % sets automatic line breaking
  captionpos=b,                    % sets the caption-position to bottom
  commentstyle=\color{mygreen},    % comment style
  deletekeywords={...},            % if you want to delete keywords from the given language
  escapeinside={\%*}{*)},          % if you want to add LaTeX within your code
  extendedchars=true,              % lets you use non-ASCII characters; for 8-bits encodings only, does not work with UTF-8
  frame=single,                    % adds a frame around the code
  keepspaces=true,                 % keeps spaces in text, useful for keeping indentation of code (possibly needs columns=flexible)
  keywordstyle=\color{blue},       % keyword style
  language=Octave,                 % the language of the code
  morekeywords={*,...},            % if you want to add more keywords to the set
  numbers=left,                    % where to put the line-numbers; possible values are (none, left, right)
  numbersep=5pt,                   % how far the line-numbers are from the code
  numberstyle=\tiny\color{mygray}, % the style that is used for the line-numbers
  rulecolor=\color{black},         % if not set, the frame-color may be changed on line-breaks within not-black text (e.g. comments (green here))
  showspaces=false,                % show spaces everywhere adding particular underscores; it overrides 'showstringspaces'
  showstringspaces=false,          % underline spaces within strings only
  showtabs=false,                  % show tabs within strings adding particular underscores
  stepnumber=2,                    % the step between two line-numbers. If it's 1, each line will be numbered
  stringstyle=\color{mymauve},     % string literal style
  tabsize=2,                       % sets default tabsize to 2 spaces
  title=\lstname                   % show the filename of files included with \lstinputlisting; also try caption instead of title
}

%opening
\title{Übungsblatt 6}
\author{Uli Köhler (10580373), Tobias Harrer (10575835)}
\begin{document}

\maketitle
\section*{Aufgabe 1}
Präambel: Für ein Zeichen $x \in \Sigma$ sei $x^n$ für $n \in \mathbb{N}$ definiert als die n-malige Wiederholung von $x$.\\
Die Familie F sei nun definiert durch die Suchmustermenge $s_n$ und die Textmenge $t_n$, wobei $o := max(m,n)$	\\
Definition des Suchmusters: $s_n = ab^{o-1}$\\
Definition des Textes: $t_n = b^{10\cdot o}$\\
Daraus folgt $c = 0.9$\\
Da der Mismatch immer an der ersten Stelle von s auftritt, kann jeweils nur um 1 geshiftet werden.
Da $|s| = 0.1*|t|$ gilt, und Boyer-Moore um $|s|$ weniger shifted gilt $c = 0.9$
\section*{Aufgabe 2}
Ausgehend von der Aufgabenstellung sei definiert $k := max(n,m)$\\
sowie $s := b^k$\\
Ferner sei der Text $t$ wie Folgt definiert $t := {ab^{k-1}}^p$ für $p >= 1; p \in \mathbb{N}$\\
Daraus folgt, dass die Anzahl der Vergleiche exakt $ = k + 2 * \sum_{i=1}^{k} i = k+k(k+1)=2k+k^2 >= c * (|t|*|s|) > (c * 3k^2) $ mit $c = \frac{1}{3}$, q.e.d


\section*{Aufgabe 3}%T
Die \textit{extended bad character Rule} Tabelle wird als zweidimensionales int Array der Größe $m\cdot |\Sigma|$ dargestellt. Das
Alphabet wird der Funktion übergeben. Anmerkung: Sei ein Mismatch an Position $j\in [0:m-1]$ in $s$ mit
$t_{i+j} = a$, dann muss $ebc[j,a']$ aufgerufen werden, wobei gilt $a=\Sigma[a']$.\newline
\begin{algorithmic}
\Function{compute\_ebc}{int[][] $ebc$, int $m$, char[] $s$, char[] $\Sigma$}
\State $ebc[s][length(\Sigma)]$;
\State $ebc[0][length(\Sigma)] := -1$;//Vor $s_0$ kann kein $c\in \Sigma$ vorkommen
\For{(int $i := 1$; $i< m$; $i ++$)}
  \For{(int $j := 0$; $j < length(\Sigma)$; $j ++$)}
    \If{($s[i-1]==\Sigma[j]$)}
      \State $ebc[i][j] := i-1$;
    \Else
      \State $ebc[i][j] := ebc[i-1][j]$;
    \EndIf
  \EndFor
\EndFor
\EndFunction
\end{algorithmic}

\begin{itemize}
 \item \textit{Laufzeit:} Die Initialisierung von $ebc[][]$ in der ersten Zeile benötigt genau $m\cdot |\Sigma|$ Operationen. In der zweiten Zeile
 wird die Spalte mit Index 0 von $ebc[][]$ initialisiert, was $|\Sigma|$ Operationen benötigt. In den beiden folgenden for-Schleifen werden
 mit der Überprüfung der Bedinung der if-Abfrage und den Array-Zuweisungen, von denen entweder die im if- bzw. die im else-Teil ausgeführt
 wird nur Operationen mit konstanter Laufzeit ausgeführt. Die beiden for-Schleifen laufen insgesammt $|\Sigma| \cdot (m-1)$ mal. Insgesamt
 werden also $2\cdot m\cdot |\Sigma|\in \mathcal{O}(m\cdot |\Sigma|)$ Operationen durchgeführt.
 \item \textit{Korrektheit:} Der Algorithmus betrachtet der Reihe nach alle $s_0,s_1,...,s_{m-2},s_{m-1}$. Für den ersten Buchstaben in $s$,
 also $s_0$, wird in der ersten Spalte von $ebc$ die Zahl $-1$ eingetragen, denn vor $s_0$ kann kein anderer Buchstabe vorkommen.
 Für alle anderen Buchstaben wird über das Alphabet iteriert und geprüft, ob der gerade betrachtete Buchstabe $s_i$ gleich dem $j$-ten Buchstaben
 von $\Sigma$ ist:
  \begin{itemize}
   \item Fall ($s[i-1]==\Sigma[j]$): Dann ist das das rechteste Vorkommen von $a:=\Sigma[j]$ in $s_0...s_{i-1}$ an der Stelle $s_{i-1}$, also ist
   $ebc[i][j] = i-1$.
   \item Fall ``else'': ($s[i-1]\neq \Sigma[j]$): Dann ist das rechteste Vorkommen von $a:=\Sigma[j]$ in $s_0...s_{i-1}$ das rechteste Vorkommen
   von $a$ in in $s_0...s_{i-2}$, also kann der Wert von $ebc[i-1][j]$ übernommen werden. Folglich ist $ebc[i][j] = ebc[i-1][j]$. Ist $a$ bis zur
   Position $i-1$ noch nie vorgekommen, so wird $-1$ aus der ersten Spalte übernommen.
  \end{itemize}
  Da die gesamte $ebc$-Tabelle auf diese Weise gefüllt wird, gibt sie für jedes $a\in \Sigma$ und $j\in [0:m-1]$ den korrekten Shift gemäß der
  \textit{extended bad character Rule} Definition wieder.
\end{itemize}


\section*{Aufgabe 4}%T
Laut Lemma 2.24 können für ein Wort $s\in \Sigma^{m'}$ die Z-Werte $Z_i$ für $i\in [1 : m'-1]$ mit
maximal $2m'-2$ Zeichenvergleichen bestimmt werden. In diesem Fall ist $m'=m+n+1$, also werden maximal $2\cdot(m+n+1)-2 = 2\cdot(m+n)$ Vergleiche
ausgeführt. Nun betrachten wir alle Z-Werte $Z_i$ und merken uns die Werte $i-m-1$, sodass $Z_i=m$, denn diese Werte sind die Startindezes für alle
Vorkommen von $s$ in $t$. Dies gilt, da $m$ ein Präfix von $s\$t$ ist und dann: $(Z_i = m) \Leftrightarrow (s_i...s_{i+Z_i-1} = s_0...s_{m-1} = s)$.
Dies erfordert nochmals $n+m+1$ Operationen. Insgesamt kommt man auf $2\cdot(m+n)+m+n+1 = 3\cdot(m+n)+1\in \mathcal{O}(n+m)$ Zeit, um festzustellen,
wo überall $s$ in $t$ als Teilwort auftritt. (http://www.cs.umd.edu/class/fall2011/cmsc858s/Lec02-zalg.pdf)
\end{document}
