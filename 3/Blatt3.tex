\documentclass[a4paper,10pt]{scrartcl}
\usepackage[utf8]{inputenc}
\usepackage{listingsutf8}
\usepackage[pdftex]{graphicx}
\usepackage[ngerman]{babel}
\usepackage{url}
\usepackage{hyperref}
\usepackage{amsfonts}
\usepackage{amssymb}
\usepackage{amsmath}
\usepackage{tikz}
\usepackage{textcomp}
\usepackage{amsmath}
\usepackage{lipsum}%%%%%%%%%%%%%%%%%%%%%%%%%%%%%%%%%%%%%%%%%%%%%%%%%%%%%%%%%%%%%%%%%%%%%%%%%%%%%%%%%%%%
\usepackage{mathtools}
\usepackage{amsfonts}
\usepackage{amssymb}
\usepackage{amsmath}
\usepackage{booktabs}
%usepackage[utf8x]{inputenc}
\usepackage[T1]{fontenc}
\usepackage{lmodern} %Latin modern = enhanced CM font
\usepackage{xspace} %Space enhancements
\usepackage{algorithm}%%%%%%%%%%%%%%%%%%%%%%%%%%%%%%%%%%%%%%%%%%%%%%%%%%%%%%%%%%%%%%%%%%%%%%%%%%%%%%%%%%
\renewcommand{\thefootnote}{\fnsymbol{footnote}}

%opening
\title{Übungsblatt 3}
\author{Uli Köhler (10580373), Tobias Harrer (10575835)}
\begin{document}

\newcommand{\Olog2n}{\ensuremath\mathcal{O}(log_2(n))}%%%%%%%%%%%%%%%%%%%%%%%%%%%%%%%%%%%%%%%%%%%%%%%%%%%
\maketitle

\section*{Aufgabe 1}

\begin{align*}
  %\numberthis\log_{c}(n) = \frac{\log_{2}{n}}{\log_{2}{c}} = \frac{1}{\log_2(c)}\log_2(n) \in \ \text{q.e.d}\\
  %\numberthis\text{Gegenbeispiel: Wähle } f(n) &= 2\log_2(n), \quad g(n) &= \log_2(n) \rightarrow\\
  %2\log_2(n) &\in {\mathcal O}(\log_2(n)) \nRightarrow 2^{f}=2^{2\log_2(n)}=2^{\log_2(n^2)}=n^2 \notin {\mathcal O}(n) = {\mathcal O}(2^{\log_2(n)}) = {\mathcal O}(2^{log_2(n)}) \Rightarrow \lightning
  \end{align*}
\section*{Aufgabe 2}
\section*{Aufgabe 3}
Induktionsanfang, n=1:\newline
$T(1) = 0 \in \mathcal{O}(1) \subset \mathcal{O}(n) \subset \mathcal{O}(n\cdot log^2 n)$, da der Logarithmus eine streng monoton wachsende
Funktion ist, kann auch sein Quadrat multipliziert mit einer linearen Funktion diese nur stärker wachsen lassen.\newline
Induktionsschritt, $n\rightarrow 2\cdot n$:\newline
$T(2\cdot n) = 2\cdot T(2\cdot n/2)+n\cdot log(n) = 2\cdot \underbrace{T(n)}_{\in \mathcal{O}(n\cdot log^2 n)\newline nach I.-V.}+n\cdot log(n)
\leq 2\cdot T(n) + \underbrace{n\cdot log^2 n}_{\in \Theta(n\cdot log^2 n)}\in \mathcal{O}(n\cdot log^2 n)$
\section*{Aufgabe 4}
\end{document}
