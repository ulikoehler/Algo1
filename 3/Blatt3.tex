\documentclass[a4paper,10pt,oneside,leqno]{scrartcl}
\usepackage[utf8]{inputenc}
\usepackage{listingsutf8}
\usepackage[pdftex]{graphicx}
\usepackage[ngerman]{babel}
\usepackage{url}
\usepackage{hyperref}
\usepackage{amsfonts}
\usepackage{amssymb}
\usepackage{amsmath}
\usepackage{tikz}
\usepackage{textcomp}
\usepackage{amsmath}
%\usepackage{lipsum}%%%%%%%%%%%%%%%%%%%%%%%%%%%%%%%%%%%%%%%%%%%%%%%%%%%%%%%%%%%%%%%%%%%%%%%%%%%%%%%%%%%%
\usepackage{mathtools}
\usepackage{amsfonts}
\usepackage{amssymb}
\usepackage{amsmath}
\usepackage{booktabs}
%usepackage[utf8x]{inputenc}
\usepackage[T1]{fontenc}
\usepackage{lmodern} %Latin modern = enhanced CM font
\usepackage{xspace} %Space enhancements
%\usepackage{algorithm}%%%%%%%%%%%%%%%%%%%%%%%%%%%%%%%%%%%%%%%%%%%%%%%%%%%%%%%%%%%%%%%%%%%%%%%%%%%%%%%%%%
\renewcommand{\thefootnote}{\fnsymbol{footnote}}
\newcounter{subeq}
\renewcommand{\thesubeq}{\theequation\arabic{subeq}}
\newcommand{\subeqnblock}{\setcounter{subeq}{0}\refstepcounter{equation}}
\newcommand{\subeqn}{\refstepcounter{subeq}}

%opening
\title{Übungsblatt 3}
\author{Uli Köhler (10580373), Tobias Harrer (10575835)}
\begin{document}

\maketitle

\section*{Aufgabe 1}
\renewcommand{\theequation}{\alph{equation}}
\begin{align}
  \log_{c}(n) &= \frac{\log_{2}{n}}{\log_{2}{c}} = \log_2(n)\cdot\overbrace{\frac{1}{\log_2(c)}}^{\text{Konstant, da c konst.}} \in \mathcal{O}(log_2(n))\ q.e.d\\
  f(n) &:= 2\log_2(n)\ \text{sowie}\ g(n) := \log_2(n) \rightarrow\\
  &2\log_2(n) \in \mathcal{O}(\log_2(n)) \rightarrow 2^{f}\nonumber\\
  &=2^{2\log_2(n)}=2^{\log_2(n^2)}=n^2 \notin \mathcal{O}(n) = \mathcal{O}(2^{\log_2(n)}) = \mathcal{O}(2^{log_2(n)})\nonumber
\end{align}
Beweis a) siehe (1), Beweis b) (durch Gegenbeispiel) siehe (2)
\section*{Aufgabe 2} \setcounter{equation}{1}
Anwendung des Master-Theorems (wenn anwendbar) nach Skript Heun, 2012\\
\begin{align*}
  a): T(n) &= 9 \cdot T(n/3) + n^2 \rightarrow\\
     n^2 &\notin \mathcal{O}(n^{2-e}) = \mathcal{O}(n^{\log_3(9)-e}) = \mathcal{O}(n^{\log_b(a)-e}), \forall e > 0 \rightarrow\\
     n^2 &\in \Theta(n^2) = \Theta(n^{\log_3(9)}) = \Theta(n^{\log_b(a)}) \rightarrow\\
     T(n) &\in \Theta(n^{\log_b(a)}\log(n)) = \Theta(n^{\log_3(9)}\log(n)) = \Theta(n^2\log(n))\\
     \rightarrow a &= 9, \quad b = 3, \quad f(n) = n^2\\
%%%%%%%%%%%%%%%%%%%%%%%%%%
\end{align*}
\begin{align*}
  b)T(n)&= 3 \cdot T(n/2) + n \log(n)\\
   n \log(n) &\in \mathcal{O}(n^{\log_2(3)-e})\\
   &= \mathcal{O}(n^{\log_b(a)-e})\\
   \text{ für } e &= 0.08 > 0 \\
   \rightarrow T(n) &\in \Theta(n^{\log_b(a)}) = \Theta(n^{\log_2(3)})\\
  n \log(n) &\in \mathcal{O}(n^{1.5}) \text{, da gilt:}\\
    \lim_{n \to \infty} \frac{n \log(n)}{n^{1.5}} &\overbrace{\rightarrow}^{LHospital}\lim_{n \to \infty} \frac{(n)' \log(n) + n (\log(n))'}{\frac{3}{4} \cdot n^{\frac{1}{2}}}\\
    &= \lim_{n \to \infty} \frac{\log(n)+1}{\frac{3}{4} \cdot n^{\frac{1}{2}}}
     \underset{\text{LHospital}}{\rightarrow}& \lim_{n \to \infty} \frac{\frac{1}{n}}{1.5 \cdot 0.5 \cdot n^{-0.5}} \\
     &=\lim_{n \to \infty} \frac{4\cdot1}{3\cdot n}\sqrt{n} = \lim_{n \to \infty} \frac{4}{3\sqrt{n}} = 0\\
   %%%%%%%%%%%%%%%%%%%%%%%%
\end{align*}
\begin{align*}
   c): T(n)&= 5 \cdot T(n/3) + n^2\\
    &\rightarrow a = 5, \quad b = 3, \quad f(n) = n^2\\
    n^2 &\notin \mathcal{O}(n^{1.464...-e}) = \mathcal{O}(n^{\log_3(5)-e}) = \mathcal{O}(n^{\log_b(a)-e}), \forall e > 0 \\
    n^2 &\notin \Theta(n^{1.464...}) = \Theta(n^{\log_3(5)}) = \Theta(n^{\log_b(a)}) \rightarrow \text{n.a.}\\
    n^2 \in \Omega(n^2) &= \Omega(n^{\log_3(5)+e}) = \Omega(n^{\log_b(a)+e}) \text{für ein} e = 2 - \log_3(5) \approx 0.535 > 0 \wedge\\
    a \cdot f(\frac{n}{b}) &= 5 \cdot f(\frac{n}{3}) = 5 \cdot (\frac{n}{3})^2 = \frac{5 \cdot n^2}{9} \leq \frac{5}{9} \cdot n^2\\
    &= c \cdot f(n) \text{mit c } = \frac{5}{9} < 1 \rightarrow\\
    T(n) \in \Theta(n^2)\\
   %%%%%%%%%%%%%%%%%%%%%%%%%%%%%%%%%%%%%%%%%%%%
\end{align*}
\begin{align*}
   d): T(n)&= 3 \cdot T(n/3) + n \log(n)\\
    &\rightarrow a = 3, \quad b = 3, \quad f(n) = n \log (n) \\
    n \log(n) &\notin \mathcal{O}(n^{1-e}) = \mathcal{O}(n^{\log_3(3)-e})\\
      &= \mathcal{O}(n^{\log_b(a)-e}), \forall e > 0 \rightarrow \text{n.a.}\\
    n \log(n) &\notin \Theta(n) = \Theta(n^1) = \Theta(n^{\log_3(3)})\\
      &= \Theta(\log_b(a)) \rightarrow \text{n.a.}\\
    n \log(n) &\notin \Omega(n^{1+e}) = \Omega(n^{log_1(1)+e}) = \Omega(n^{log_b(a)+e}) \forall e > 0\\
    \text{Sei } n &= 3^k \iff k = \log_3(n) \iff\\
    T(n) &= c \cdot T(n/3) + n \log n = \sum_{i=0}^{k} c 3^k \log(3^{k-i})\\
    &= c\cdot3^{k} \sum_{i=0}^{k} \log(3^{k-i})\\
    &= c\cdot3^k \sum_{i=0}^{k}(k-i)\log(3)\\
    &= c\cdot3^k\log(3) \sum_{i=0}^{k}(k-i)\\
    &= c\cdot3^k\log(3) \left[\sum_{i=0}^{k}k-\sum_{i=0}^{k}i\right]\\
    &= c\cdot3^k\log(3) \left[ (k+1)k - \frac{k(k+1)}{2} \right]\\
    &= c\cdot3^k\log(3) \frac{k(k+1)}{2}\\
    &= c\cdot3^k\log(3) \frac{k^2+k}{2}\\
    &= c\cdot3^{\log_3(n)}\log(3)\frac{1}{2}((\log_3(n))^2+\log_3(n))\\
    &= \underbrace{c\log(3)\frac{1}{2}}_{\text{= Konstante c'}}n((\log_3(n))^2+\log_3(n))\\
    &= c' n((\log_3(n))^2+\log_3(n)) = c' (n(\log_3^2(n))+n\log_3(n)) \in \Theta(n \log^2(n))
\end{align*}
\section*{Aufgabe 3}
Induktionsanfang, n=1:\newline
$T(1) = 0 \in \mathcal{O}(1) \subset \mathcal{O}(n) \subset \mathcal{O}(n\cdot log^2 n)$, da der Logarithmus eine streng monoton wachsende
Funktion ist, kann auch sein Quadrat multipliziert mit einer linearen Funktion diese nur stärker wachsen lassen.\newline
Induktionsschritt, $n\rightarrow 2\cdot n$:\newline
$T(2\cdot n) = 2\cdot T(2\cdot n/2)+n\cdot log(n) = 2\cdot \underbrace{T(n)}_{\in \mathcal{O}(n\cdot log^2 n)\newline nach I.-V.}+n\cdot log(n)
\leq 2\cdot T(n) + \underbrace{n\cdot log^2 n}_{\in \Theta(n\cdot log^2 n)}\in \mathcal{O}(n\cdot log^2 n)$\newline
Also gilt $\forall n \geq 1: T(n) \in \mathcal{O}(n\cdot log^2 n) \Leftrightarrow T(2\cdot n) \in \mathcal{O}(n\cdot log^2 n)$.\newline
Da sowohl alle Funktionen in $\mathcal{O}(n)$, wie auch in $\mathcal{O}(n\cdot log(n))$ und $\mathcal{O}(n\cdot log^2 n)$ streng monoton wachsend,
sind, gilt dies auch für alle $n'$, die zwischen $n\leq n' \leq 2\cdot n$ liegen.
\section*{Aufgabe 4}
(int*int) Summe(int[] feld, int x)\{\newline
int i = lengthOf(feld) - 1;\newline
while(x <= feld[i])\{\newline
i--;//Elemente des Felds, die größer sind als die Summe können ausgeschlossen werden.\newline
\}\newline
if(x == feld[i])\{return (i,i)\}//die ``triviale'' Lösung, ein einziges Element gleich x.\newline
int summe = 0, j = 0;\newline
for(j = 0; j <= i AND summe <= x;j++)\{//addiere sequentiell die Array Element vom nullten bis zum max. i-ten\newline
summe += feld[j];\newline
\}\newline
if(summe == x)\{return (0, j);\}//Summe geht von $a_0$ bis $a_j$\newline
else if(j == i)\{return (0,-1);\}//Da feld[i] > x, wird summe+=feld[i] größer als x sein, (0,-1) wird als ``[i,j] nicht vohanden'' definiert.\newline
for(int k = j, l = 0; k < i; k++)\{//Wiederhole (i-j+1)-mal, also von der momentanen (j) bis zur max. sinnvollen Position (i)\newline
while(summe > x)\{\newline
summe-=feld[l];//Ziehe von summe so lange die kleinsten Elemente des Felds ab, bis summe >= x\newline
l++;\newline
if(x == summe)\{return (l,k);\}\newline
\}\newline
summe+=feld[k];//Zähle dann das nächst größere Element (feld[k]) dazu.\newline
if(x == summe)\{return (l,k);\}\newline
\}\newline
return (0, -1);//Kommt der Algorithmus bis hier, gibt es kein Intervall [i,j], s.d. $a_i+...+a_j = x$ erfüllt wäre.\newline
\}\newline
Alle for- und while-Schleifen werden maximal n (= Länge des Felds) mal durchlaufen. Die Anzahl dieser Schleifen ist nicht von der Eingabegröße
abhängig, sondern liegt konstant bei vier Schleifen. In jeder Schleife werden lediglich Additionen (bzw. Subtraktionen) ausgeführt oder Zugriffe
auf Arrays, die jew. in konstanter Zeit ablaufen. Auch die letzte for-Schleife, in der sich eine weitere while-Schleife befindet, führt maximal n
konstante Zeit benötigende Operationen aus. Dadurch liegt die Gesamtlaufzeit in $\mathcal{O}(n)$.\newline
Falls es eine Lösung gibt, wird diese auch vom Algorithmus gefunden. Falls sie ein einzelnes Element ist, wird die Lsg. in Z.6 gefunden.
Falls die Lsg. bei [0,j] liegt, wird sie in Z. 12 gefunden. Liegt sie in [i,j], dann wird sie in Z.21 oder Z.24 gefunden. Kommt der Algorithmus
bis Zeile 15, ist die Lsg. [i,j], oder sie ex. nicht. Angenommen sie existiert, dann ist in Z.15 summe > x. Nun wird versucht x==summe
herbeizuführen, indem die ersten i Elemente des Felds wieder abgezogen werden, bis summe<=x. Falls summe<x, wird das nächst größere Element
des Feldes hinzuaddiert und die kleinsten verbliebenen Elemente des Felds wieder abgezogen. Wiederholt man dieses Verfahren, bis man das
Feldelement hinzuaddiert, das > x ist. Hat man eine Lösung gefunden, sonst existiert keine. Des weiteren gilt: Falls beim addieren der 
kleinsten Feldelemente das i-te erreicht wurde, das selbst > x ist, muss summe + feld[i] > x sein, und es gibt keine Lösung [i,j].
\end{document}
