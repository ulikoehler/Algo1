\documentclass[a4paper,10pt,oneside,leqno]{scrartcl}
\usepackage[utf8]{inputenc}
\usepackage{listingsutf8}
\usepackage[pdftex]{graphicx}
\usepackage[ngerman]{babel}
\usepackage{url}
\usepackage{hyperref}
\usepackage{amsfonts}
\usepackage{amssymb}
\usepackage{amsmath}
\usepackage{tikz}
\usepackage{listings}
\usepackage{textcomp}
\usepackage{amsmath}
%\usepackage{lipsum}%%%%%%%%%%%%%%%%%%%%%%%%%%%%%%%%%%%%%%%%%%%%%%%%%%%%%%%%%%%%%%%%%%%%%%%%%%%%%%%%%%%%
\usepackage{mathtools}
\usepackage{amsfonts}
\usepackage{amssymb}
\usepackage{amsmath}
\usepackage{booktabs}
%usepackage[utf8x]{inputenc}
\usepackage[T1]{fontenc}
\usepackage{lmodern} %Latin modern = enhanced CM font
\usepackage{xspace} %Space enhancements
%\usepackage{algorithmic}%%%%%%%%%%%%%%%%%%%%%%%%%%%%%%%%%%%%%%%%%%%%%%%%%%%%%%%%%%%%%%%%%%%%%%%%%%%%%%%%%%
\usepackage{algpseudocode}
\renewcommand{\thefootnote}{\fnsymbol{footnote}}
\definecolor{mygreen}{rgb}{0,0.6,0}
\definecolor{mygray}{rgb}{0.5,0.5,0.5}
\definecolor{mymauve}{rgb}{0.58,0,0.82}
\lstset{ %
  backgroundcolor=\color{white},   % choose the background color; you must add \usepackage{color} or \usepackage{xcolor}
  basicstyle=\footnotesize,        % the size of the fonts that are used for the code
  breakatwhitespace=false,         % sets if automatic breaks should only happen at whitespace
  breaklines=true,                 % sets automatic line breaking
  captionpos=b,                    % sets the caption-position to bottom
  commentstyle=\color{mygreen},    % comment style
  deletekeywords={...},            % if you want to delete keywords from the given language
  escapeinside={\%*}{*)},          % if you want to add LaTeX within your code
  extendedchars=true,              % lets you use non-ASCII characters; for 8-bits encodings only, does not work with UTF-8
  frame=single,                    % adds a frame around the code
  keepspaces=true,                 % keeps spaces in text, useful for keeping indentation of code (possibly needs columns=flexible)
  keywordstyle=\color{blue},       % keyword style
  language=Octave,                 % the language of the code
  morekeywords={*,...},            % if you want to add more keywords to the set
  numbers=left,                    % where to put the line-numbers; possible values are (none, left, right)
  numbersep=5pt,                   % how far the line-numbers are from the code
  numberstyle=\tiny\color{mygray}, % the style that is used for the line-numbers
  rulecolor=\color{black},         % if not set, the frame-color may be changed on line-breaks within not-black text (e.g. comments (green here))
  showspaces=false,                % show spaces everywhere adding particular underscores; it overrides 'showstringspaces'
  showstringspaces=false,          % underline spaces within strings only
  showtabs=false,                  % show tabs within strings adding particular underscores
  stepnumber=2,                    % the step between two line-numbers. If it's 1, each line will be numbered
  stringstyle=\color{mymauve},     % string literal style
  tabsize=2,                       % sets default tabsize to 2 spaces
  title=\lstname                   % show the filename of files included with \lstinputlisting; also try caption instead of title
}

%opening
\title{Übungsblatt 7}
\author{Uli Köhler (10580373), Tobias Harrer (10575835)}
\begin{document}

\maketitle
\section*{Aufgabe 1}
Siehe Zeichnung auf Blatt 1
\section*{Aufgabe 2}
Siehe Zeichnung auf Blatt 2.1 und 2.2
\section*{Aufgabe 3}
Siehe Zeichnung Blatt 3
\section*{Aufgabe 4}
Gegeben $w \in \Sigma^m$
  \subsection*{a)}
  Alle Z-Werte positiv: dann ist $w = c^m$ für ein festes, aber beliebiges $c \in \Sigma$, d.h. $w$ ist z.B. ``$\underbrace{aaaa...aaa
  }_{m-mal}$''. Dann sind alle Z-Werte $Z_i$ gleich $m-i$, was echt größer 0 ist, da $i \leq m-1$. Angenommen ein $w \in \Sigma^m$ besteht
  aus mehr als einem einzigen, sich wiederholenden $c \in \Sigma$, z.B. ``aaabaaaa'', dann ist $Z_3=0$, da $w_0 = a \neq  b=w_3$. Es ist
  dabei unerheblich, wo und wie oft ein unterschiedlicher Buchstabe eingefügt wird, der Z-Wert wird mindestens für diese Stelle 0 und ist
  somit nicht echt positiv.
  \subsection*{b)}
  Genau ein Z-Wert ist 0: Hier lässt sich das obige Beispiel ``aaa...aabaaa...aaaa'' verwenden, in dem $w = c^+c'c^+ | c^+c'$ mit
  $c$, $c' \in \Sigma^m$ und $c \neq c'$ ist. Also entweder wie oben, oder ``aaa...aab''. Wenn $c'$ an der Stelle
  $w_i$ steht, dann ist $Z_i=0$, und nur an dieser Stelle, da an jeder anderen Position mindestens $w_0 = a = w_j$ gilt, für $j \neq i$.
  Angenommen an einer anderen Position $j \neq i$ und $j \neq 0$ stünde ``b'', dann wäre $Z_j$ ebenfalls gleich 0. Also darf es nur eine
  einzige Position $i>0$ und $i<m$ geben, an der ``b'' bzw. allgemein $c'$ steht, wenn an allen anderen Positionen ``a'' bzw. $c$ steht.
\end{document}
