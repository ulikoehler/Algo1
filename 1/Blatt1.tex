\documentclass[a4paper,10pt]{article}
\usepackage[utf8]{inputenc}
\usepackage{listingsutf8}
\usepackage[pdftex]{graphicx}
\usepackage[ngerman]{babel}
\usepackage{url}
\usepackage{hyperref}
\usepackage{amsfonts}
\usepackage{amssymb}
\usepackage{amsmath}
%usepackage[utf8x]{inputenc}
\usepackage[T1]{fontenc}
\usepackage{lmodern} %Latin modern = enhanced CM font
\usepackage{xspace} %Space enhancements

%opening
\title{Übungsblatt 1}
\author{Uli Köhler (10580373), Tobias Harrer (10575835)}

\begin{document}

\maketitle

\section{Aufgabe 2}
Idee: Erweiterung des 'Cleveren Algorithmus', sodass in der letzten Fallunterscheidung auch rMaxScores zugelassen werden,
die gleichgroß wie der bestehende maxScore sind. Die Start- und Endpositionen jedes weiteren maxScore werden in einer Liste gespeichert:
\newline
\begin{tabbing}
MSS \= Clever (int[] a, int n)\\
begin\\
\>int maxscore   := 0;\\
\>int rmaxscore  := 0; rstart := 1;\\
\>\textbf{int[] lStart; int[] rEnd; int MSScounter := 0;}\\
\>for \= (i := 1; i $\leq$ n; i++) do\\
\> \>if \=(rmaxscore > 0) then\\
\> \> \>rmaxscore := rmaxscore + a[i];\\
\> \> else \= \\
\> \> \>rmaxscore := a[i]; rstart := i;\\
\> \>\textbf{if (rmaxscore $\geq$ maxscore) then}\\
\> \> \> maxscore := rmaxscore; \\ \> \> \> \textbf{lStart[MSScounter] := rstart;}\\ \> \> \> \textbf{rEnd[MSScounter] := i;}\\
\> \> \>  \textbf{MSScounter++;}\\
end 
\end{tabbing}

\section{Aufgabe 3}
 \begin{tabular}{lccccr}
   & 1s & 60s & 3,600s & 86,400s & 2,592,000s \\
  T1 & 2 &120 &	7,200 &	172,800 &5,184,000 \\
  T2&$\approx$7&$\approx$164&$\approx$5,763&$\approx$103,700 &$\approx$2,445,000\\
  T3&$\approx$32&$\approx$245&$\approx$1,897&$\approx$9,295&$\approx$50,912\\
  T4&$\approx$46&$\approx$182&$\approx$711&$\approx$2,052&$\approx$6,376\\
  T5&$\approx$13&$\approx$16&$\approx$20&$\approx$23&$\approx$26\\
 \end{tabular}

\end{document}