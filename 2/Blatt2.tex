\documentclass[a4paper,10pt,freqn]{article}
\usepackage[utf8]{inputenc}
\usepackage{listingsutf8}
\usepackage[pdftex]{graphicx}
\usepackage[ngerman]{babel}
\usepackage{url}
\usepackage{hyperref}
\usepackage{amsfonts}
\usepackage{amssymb}
\usepackage{tikz}
\usepackage{textcomp}
\usepackage{amsmath}
\usepackage{booktabs}
%usepackage[utf8x]{inputenc}
\usepackage[T1]{fontenc}
\usepackage{lmodern} %Latin modern = enhanced CM font
\usepackage{xspace} %Space enhancements
%\usepackage{algorithm}
%\renewcommand{\thefootnote}{\fnsymbol{footnote}}
%opening
\title{Übungsblatt 2}
\author{Uli Köhler (10580373), Tobias Harrer (10575835)}
\begin{document}

\maketitle

\section*{Aufgabe 1}
Binäre Suche:\newline

boolean FIND(int[] array, int wanted) \{\newline
        int curPos = (int) Math.ceil(array.length / 2);\newline
        for (int i = 0; i <= Math.ceil(Math.log(array.length) / Math.log(2)); i++) \{\newline
            if (array[curPos] == wanted) \{\newline
                return true;\newline
            \} else if (wanted < array[curPos]) \{\newline
                curPos -= (int) Math.ceil(array.length / (Math.pow(2, i + 2)));\newline
            \} else \{\newline
                curPos += (int) Math.ceil(array.length / (Math.pow(2, i + 2)));\newline
            \}\newline
            curPos = (curPos < 0) ? 0 : curPos;\newline
            curPos = (curPos >= array.length) ? array.length - 1 : curPos;\newline
        \}\newline
        return false;\newline
    	\}\newline
    	
\begin{bfseries}Eingabegröße:\end{bfseries}\newline
Die Eingabegröße hängt von  der zum einen von der Größe des Arrays ab und von der
Beschaffenheit der Array-Elemente, z.B. benötigen i.A. n Zahlen weniger Speicher
als n Strings.

\begin{bfseries}Charakteristische Operation:\end{bfseries}\newline
Die charakteristische Operation ist der Vergleich eines beleibigen Elements $A_i$ des Arrays mit dem gesuchten $x$.
Der Aufruf von $A_i$ und der Vergleich benötigen jeweils konstante Zeit. Diese Operationen werden maximal
$\lceil log_{2}(n)\rceil$ mal durchgeführt, wobei $n$ die Länge des Arrays ist.

\begin{bfseries}Korrektheit:\end{bfseries}\newline
Durch bis zu $\lceil log_{2}(n)\rceil$-maliges ``halbieren'' des Arrays kann jedes Element, das darin enthalten ist,
gefunden werden. Dies folgt daraus, dass die Menge gemäß einer totalen Ordnung sortiert ist. Bei jeder Halbierung wird
das Intervall des Aufenthaltsorts des gesuchten Elements weiter eingeschränkt. Es kann sich nicht außerhalb der
jeweils gewählten Hälfte befinden, da die <-Relation dies nicht zulässt. Ist ein Element nach $\lceil log_{2}(n)\rceil$
Suchschritten nicht gefunden, so ist es auch nicht in der Menge enthalten. $\lceil log_{2}(n)\rceil$ beschreibt dabei
die maximale Anzahl, wie oft eine Zahl halbiert werden kann, sodass eine Zahl $0<n\leq1$ herauskommt. In dem Fall hat
das Intervall, auf das der Aufenthaltsort eingeschränkt wurde, die Länge 1. Falls sich das gesucht Element dort nicht
befindet, ist es folglich nicht Teil der Menge.

\begin{bfseries}Laufzeit:\end{bfseries}\newline
Da die Eingabegröße $n$ endlich ist, ist auch die Anzahl der Schleifendurchläufe $= log_{2}(n)$ endlich,
und der Algorithmus wird terminieren. Da mit dem Arrayzugriff und dem Vergleich nur konstante Operationen
in $log_{2}(n)$ Schleifendurchläufen ausgeführt werden, liegt die Gesamtlaufzeit in $\Theta(log_{2}(n))$.

\section*{Aufgabe 2}
\subsection*{a)}
$f(n)=n\cdot3^{k} \in \Theta(n)$, da $3^k$ lediglich eine Konstante darstellt, die nicht von $n$ abhängt.

\subsection*{b)}
$\lim_{x \to \infty}f(n) = lim_{x \to \infty}\frac{n^3-n^2+5}{n^3+4n^2-3n} = \lim_{x \to \infty}\frac{n^3}{n^3} = 1 \Rightarrow
\frac{n^3-n^2+5}{n^3+4n^2-3n} \in \Theta(1)$

\subsection*{c)}
$f(n) = 4^{log_{2}(n)} = 2^{2*log_{2}(n)} = 2^{log_{2}(n) +log_{2}(n)} = 2^{log_{2}(n)}\cdot2^{log_{2}(n)} = n*n
 = n^2 \in \Theta(n^2)$
 
\subsection*{d)}
$f(n) = \sum_{i=0}^{n-1} (n-1)^3 = n^3 + (n-1)^3 + \dots +(n-(n-1))^3 = n^3 + \underbrace{n^3 + 3ni^2 - 3n^2i - i^3}_{\in \Theta(n^3)\forall
0<i<n-1} + 1 \in \Theta(n^3)$

\section*{Aufgabe 3}
\subsection*{a)}
Wahr, Mengengleichheit zwischen a,b $:= a\subseteq b \wedge a\supseteq b$\newline
$\mathcal{O}(f)\cdot \mathcal{O}(g)\supseteq \mathcal{O}(f\cdot g)$: folgt aus der Definition von $\mathcal{O}(f)\cdot \mathcal{O}(g)$\newline
$\mathcal{O}(f)\cdot \mathcal{O}(g)\subseteq \mathcal{O}(f\cdot g)$: Seien $f'\in \mathcal{O}(f)$ und $g'\in \mathcal{O}(g)$,
dann gibt es $c_f>0$ und $c_g>0$, sodass $c_f\cdot f' \in  \mathcal{O}(f)$ und $c_g \cdot g' \in  \mathcal{O}(g)$.
Dann ist $c_f\cdot f' \cdot c_g \cdot g' \in  \mathcal{O}(f\cdot g)$

\subsection*{b)}
Wahr:\newline
$log(\sum_{i=0}^k a_i\cdot n^i) = log(a_k\cdot n^k + \sum_{i=0}^{k-1} a_i\cdot n^i) \in \Theta(log(a_k\cdot n^k)) = 
\Theta(\underbrace{log(a_k)}_{konstant}+log(n^k)) \in \Theta(log(n^k)) = \Theta(\underbrace{k}_{konstant}*log(n))
\in \Theta(log(n))$\newline Bei den ``$=$''-Schritten wurden Logarithmusregeln angewandt.

\subsection*{c)}
falsch:\newline
Seien $n-1 = f(n) = g(n) \in \Theta(n)$, dann ist $|f-g| = |f(n)-g(n)| = |n-1-(n-1)| = |0| = 0 \notin \Theta(n)$

\section{Aufgabe 4}
Gegeben sei $a_n = 2a_{n-1}-a{n-2} + 1$
sowie $a_0 = 0$ und $a_1 = 1$\\
\subsection{Schritt 1: Alle Vorkommen von a auf eine Seite bringen}
\begin{align*}
a_n &= 2a_{n-1}-a{n-2}+1 \\
a_n-2a_{n-1}+a_{n-2}&=1
\end{align*}
\begin{align*}
I)\quad&a_n-2a{n-1}+a_{n-2}&=1\\
II)\quad&a_n-1-2a_{n-2}+a_{n-3}&=1\\
\hline\\
II-I)\quad&a_n-3a_{n-1}+3a_{n-2}-a_{n-3}&=0\label{(1)-(2)}
\end{align*}
Aus den gegebenen Variablen ist bekannt, dass gilt:\\$a_2 = 2a_1-a_0+1 = 2\cdot1-0+1=3$\\

Charakteristisches Polynom der Rekursionsgleichung (Bestimmung trivial):\\
$C(x) = x^3-3x^2+3x-1$\\
\textrightarrow Ch. Polynom hat 3 Nullstellen.
\subsection{Finden der Nullstellen des charakteristischen Polynoms}
Nullstelle Nr. 1 muss erraten werden: $1$\\
Polynomdivision: $(x^3-3x^2+3x-1)/(x-1)=x^2-2x+1$
Dann Anwendung der Mitternachtsformel (trivial) \textrightarrow Polynom hat dreifache Nullstelle bei 1
\subsection{Aufstellen der Formel}
\begin{eqnarray*}
 a_n = \sum_{i=0}^m\sum_{j=0}^{k_i-1} \alpha_{i,j}n^i\\
 x_i^n  =\sum_{i=0}^m\sum_{j=0}^{k_i-1} \alpha_{i,j}n^j\\
 =\alpha_{1,0}n^0+\alpha_{1,1}n^1+\alpha_{1,2}n^2+\alpha_{2,0}n^0+\alpha_{2,1}n^1+\alpha_{2,2}n^2+\alpha_{3,1}n^1+\alpha_{3,2}n^2\\=
 n^0(\alpha_{1,0}+\alpha_{2,0}+\alpha_{3,0}+n^1(\alpha_{1,1}+\alpha_{2,1}+\alpha_{3,1})+n^2(\alpha_{1,2}+\alpha_{2,2}+\alpha_{3,2})\\=
 c_1n^0+c_2n^1+c_3n^2
\end{eqnarray*}

\subsection{Einsetzen der Anfangsbedingungen}
\begin{align*}
 0&=c_1+c_2n+c_3(n^2-n) &\text{für\quad}& n = 0\rightarrow 0 = c_1+c_2\cdot0+c_3(-1\cdot0) &\rightarrow c_1& = 0\\
 1&=c_2+nc_2+c_3(n^2-n) &\text{für\quad}& n = 1\rightarrow 1 = c_2+0 &\rightarrow& c_2 = 1\\
 1&=c_2+nc_2+c_3(n^2-n) &\text{für\quad}& n = 2\rightarrow 3 = 2+c_3\cdot(2\cdot1) &\rightarrow c_3& = 0.5\\
\end{align*}
Daraus folgt: $a_n = n+\frac{n^2-n}{2}=\frac{2n+n^2-n}{2}=\frac{n^2+n}{2}$ q.e.d

\section{Aufgabe 5 a)}
\textbf{Sonderfall:} Feld ist bereits sortiert \textrightarrow In diesen Fall ist die Schranke trivialerweise durch $log_2(n)$ (binäre Suche) gegebe
\newline
Im Folgenden wird davon ausgegangen, dass das zu durchsuchende Feld nicht sortiert ist, oder nicht davon ausgegangen werden kann, dass es sortiert ist.\\
Im Worst Case benötigt die Suche mindestens $\Omega(n)$ Elemente.
Der Entscheidungsbaum wird wie folgt definiert (Start beim ersten Element des Feldes):
\begin{itemize}
 \item Überprüfe, ob das zu suchende Element mit dem aktuellen Element übereinstimmt
 \begin{itemize}
  \item Wenn die Elemente übereinstimmen, terminiere (gefunden)
  \item Wenn die Elemente nicht übereinstimmen:
  \begin{itemize}
   \item Fahre fort mit dem nächsten Element
   \item Sofern kein nächstes Element existiert, terminiere (nicht gefunden)
  \end{itemize}
 \end{itemize}
\end{itemize}
Dieses Baumelement enthält genau einen Vergleich (Elementaroperation).
Worst-Case-Szenario: Das zu suchende Element kommt nicht vor, oder als letztes Element.

In diesem Fall wird der obige Anweisungsblock n mal ausgeführt, d.h. n Vergleiche werden durchgeführt.
Da $n \in \Omega(n)$ gilt, gilt die obige Behauptung q.e.d.
{\LARGE \textsc{Aber:}}
\section{Aufgabe 5 b)}
Berechnung der Wahrscheinlichkeiten über die Grundgesamtheit der Felder für ein gegebenes n:
\begin{align*}
 P(\text{Treffer an Position i}) &= \frac{1-p}{n}\\
 1-P(\text{Treffer an Position i}) &= \frac{n-1+p}{n} = 1-\left(\frac{1-p}{n}\right)\\
 P(\text{1 Vergleich benötigt}) &=  \frac{1-p}{n}\\
 P(\text{2 Vergleiche benötigt}) &=  [1-P(\text{1 Vergleich benötigt})]\cdot\frac{1-p}{n}\\
 &= np\footnotemark + \text{P(1 Vergleich benötigt) $\cdot$ (1 == Kosten für einen Vergleich)}\\&+ \text{P(2 Vergleiche benötigt) $\cdot$ (2 == Kosten für zwei Vergleiche)} + ...\\
 A(n) = E[T(L_n)] &= np+\sum_{i=1}^{n}\cdot i\left(\frac{n-1+p}{n}\right)^{i-1}\left(\frac{1-p}{n}\right)
\end{align*}
\footnotetext{Falls das zu suchende Element nicht existiert}
\section{Aufgabe 6a)}
\renewcommand{\arraystretch}{1.3} 
\begin{tabular}{ccc}
 \toprule
 n&$T_n$&$\frac{T_n}{T_{n-1}}$\\
 \midrule
 2&4&4\\
 3&13&$3.25$\\
 4&40&$\frac{40}{13}\cong3.0769$\\
 5&121&$\frac{121}{40}\cong3.025$\\
 6&364&$\frac{364}{121}\cong3.0083$\\
 7&1093&$\frac{1093}{364}\cong3.0027$\\
 8&3280&$\frac{3280}{1093}\cong3.0009$\\
 9&9841&$\frac{9841}{3280}\cong3.0003$\\
 10&29524&$\frac{29524}{9841}\cong3.0001$\\
 \bottomrule
\end{tabular}
\section{Aufgabe 6b)}
Erraten: $T(n)=\frac{3^n-1}{2}$\\
Induktionsanfang: $T(0) = \frac{3^0-1}{2} = \frac{1-1}{2} = 0 q.e.d$\\
Induktionsfortführung:
\begin{align*}
 \frac{3^n-1}{2} &= \frac{3\cdot3^{n-1}-1}{2}\\
 &= \frac{4\cdot3^{n-1}-3^n+3-4}{2}\\
 &= 2\cdot3^n-\frac{3^n}{2}+1.5-2\\
 &= 2\cdot3^n-2-1.5\cdot3^{n-1}+1.5\\
 &= 2\cdot(3^n-1)-1.5\cdot(3^{n-1}-1)\\
 &= 4\cdot(\frac{3^n-1}{2})-3\cdot(\frac{3^{n-1}-1}{2})\\
 &= 4\cdot(\frac{3^n-1}{2})-3\cdot(\frac{3^{n-1}-1}{2})\\
 &= 4T(n)-3T(n-1)\\
 &= T(n+1)\qquad q.e.d
\end{align*}

\end{document}