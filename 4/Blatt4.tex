\documentclass[a4paper,10pt,oneside,leqno]{scrartcl}
\usepackage[utf8]{inputenc}
\usepackage{listingsutf8}
\usepackage[pdftex]{graphicx}
\usepackage[ngerman]{babel}
\usepackage{url}
\usepackage{hyperref}
\usepackage{amsfonts}
\usepackage{amssymb}
\usepackage{amsmath}
\usepackage{tikz}
\usepackage{listings}
\usepackage{textcomp}
\usepackage{amsmath}
%\usepackage{lipsum}%%%%%%%%%%%%%%%%%%%%%%%%%%%%%%%%%%%%%%%%%%%%%%%%%%%%%%%%%%%%%%%%%%%%%%%%%%%%%%%%%%%%
\usepackage{mathtools}
\usepackage{amsfonts}
\usepackage{amssymb}
\usepackage{amsmath}
\usepackage{booktabs}
%usepackage[utf8x]{inputenc}
\usepackage[T1]{fontenc}
\usepackage{lmodern} %Latin modern = enhanced CM font
\usepackage{xspace} %Space enhancements
%\usepackage{algorithm}%%%%%%%%%%%%%%%%%%%%%%%%%%%%%%%%%%%%%%%%%%%%%%%%%%%%%%%%%%%%%%%%%%%%%%%%%%%%%%%%%%
\renewcommand{\thefootnote}{\fnsymbol{footnote}}
\definecolor{mygreen}{rgb}{0,0.6,0}
\definecolor{mygray}{rgb}{0.5,0.5,0.5}
\definecolor{mymauve}{rgb}{0.58,0,0.82}
\lstset{ %
  backgroundcolor=\color{white},   % choose the background color; you must add \usepackage{color} or \usepackage{xcolor}
  basicstyle=\footnotesize,        % the size of the fonts that are used for the code
  breakatwhitespace=false,         % sets if automatic breaks should only happen at whitespace
  breaklines=true,                 % sets automatic line breaking
  captionpos=b,                    % sets the caption-position to bottom
  commentstyle=\color{mygreen},    % comment style
  deletekeywords={...},            % if you want to delete keywords from the given language
  escapeinside={\%*}{*)},          % if you want to add LaTeX within your code
  extendedchars=true,              % lets you use non-ASCII characters; for 8-bits encodings only, does not work with UTF-8
  frame=single,                    % adds a frame around the code
  keepspaces=true,                 % keeps spaces in text, useful for keeping indentation of code (possibly needs columns=flexible)
  keywordstyle=\color{blue},       % keyword style
  language=Octave,                 % the language of the code
  morekeywords={*,...},            % if you want to add more keywords to the set
  numbers=left,                    % where to put the line-numbers; possible values are (none, left, right)
  numbersep=5pt,                   % how far the line-numbers are from the code
  numberstyle=\tiny\color{mygray}, % the style that is used for the line-numbers
  rulecolor=\color{black},         % if not set, the frame-color may be changed on line-breaks within not-black text (e.g. comments (green here))
  showspaces=false,                % show spaces everywhere adding particular underscores; it overrides 'showstringspaces'
  showstringspaces=false,          % underline spaces within strings only
  showtabs=false,                  % show tabs within strings adding particular underscores
  stepnumber=2,                    % the step between two line-numbers. If it's 1, each line will be numbered
  stringstyle=\color{mymauve},     % string literal style
  tabsize=2,                       % sets default tabsize to 2 spaces
  title=\lstname                   % show the filename of files included with \lstinputlisting; also try caption instead of title
}

%opening
\title{Übungsblatt 4}
\author{Uli Köhler (10580373), Tobias Harrer (10575835)}
\begin{document}

\maketitle

\section*{Aufgabe 1}
\begin{lstlisting}[language=C++]
#include <string>
#include <vector>
#include <cmath>
#include <utility>
#include <boost/lexical_cast.hpp>
/**
 * @param t The text to search in
 * @param s The string to search for
 */
std::vector<int>&& ModKMP(const std::string& t, const std::string& s) {
  std::vector<int> result;
  int border[s.size() + 1];
  border[0] = -1;
  //Loop 1
  for (int j = 2, i = 0; j < s.size() + 1; j++) {
      //Loop 1.1
      while (i >= 0 && s[i] != s[j - 1]) {
          i = border[i];
      }
      border[j] = ++i;
  }
  //Loop 2
  for(int str = 0, cmp = 0; str < t.size() - s.size() + 1;) {
       //Loop 2.1
       while(t[str + cmp] == s[cmp]) {
           if(++cmp == s.size()) {
               result.push_back(boost::lexical_cast<int>(str));
               break;
           }
       }
       str += cmp - border[cmp];
       cmp = max(0, border[cmp]);
  }
  return std::move(result);
}
\end{lstlisting}
\textbf{Terminierung:}
Der Algorithmus enthält keine Rekursion. Er terminiert daher, sofern jede der Schleifen terminiert.
Loop 1 terminiert wenn die Länge von s endlich ist, wovon auszugehen ist sowie Loop 1.1 terminiert.
Loop 1.1 terminiert, da  border[0] mit -1 initialisiert wurde, und i nach einer endlichen Zahl Iterationen
0 erreichen wird, wenn die Iteration nicht schon vorher durch s[i] != s[j - 1] abgebrochen wird.

Loop 2 terminiert wenn Loop 2.1 terminiert, da cmp - border[cmp] > 0, da border[cmp] < cmp, siehe Beweis in der Vorlesung.
Loop 2.1 terminiert, da cmp immer inkrementiert wird und daher nach endlicher Iterationszahl die Schleifenabbruchbedigungn
t[str + cmp] != s[cmp] gelten muss (spätestens wenn das Ende des Strings erreicht ist).
q.e.d
\textbf{Korrektheit}
Ausgegangen wird davon, dass die Korrektheit von KMP schon in der Vorlesung gezeigt wurde.
Da die Modifikation KMP nicht grundsätzlich modifiziert, also seine Korrektheit nicht modifiziert,
sondern nur den Zweck erweitert, muss nur gezeigt werden dass alle Vorkommen gefunden werden.
Dies wird dadurch erreicht, dass beim Finden eines Treffer nicht terminiert wird sondern der Index
des gefundenen Vorkommens zur Resultatsliste hinzugefügt wird. Damit ist das gewünschte Ziel erreicht.
\textbf{Laufzeit}
% Hier wird ebenfalls vom Laufzeitbeweis aus der Vorlesung ausgegangen.
Im Vergleich zur Vorlesung wurde der Algorithmus nur wenig verändert, insbesondere sind
keine weiteren verschachtelten Schleifen hinzugekommen.
Somit kann sich die asymptotische Komplexität nicht verändert haben, allerdings verändert sich die
Best-Case-Komplexität zum schlechteren, da selbst beim Fund des zu suchenden Strings gleich am Anfang
nicht terminiert wird sondern bis zum Ende des Textes weitergesucht werden muss.
Dadurch ergibt sich die asymptotische Laufzeitkomplexität von $\mathcal{O(n+m)}$, wie auch beim KMP.
\section*{Aufgabe 2}
\textbf{Begründungen in der Tabelle:}\\
\textbf{1:}\ da beim Einsetzen des vorherigen Borders ein Vergleich zeigt dass die jeweils folgenden Zeichen nicht gleich sind\\
\textbf{2:}\ da beim Einsetzen des vorherigen Borders ein Vergleich zeigt dass die jeweils folgenden Zeichen gleich sind\\
\begin{tabular}{l|l|p{8cm}|c}
\textbf{i} & \textbf{Präfix} & \textbf{Border} & \textbf{$BorderTabelle[i]$}\\\hline
0 &\$ & Hat nur \$ als Border & -1s\\
1 &a\$ & Border kann nicht erweitert werden & 0\\
2 &ab\$ & Border kann nicht erweitert werde & 0\\
3 &aba\$ & Border kann hier erweitert werden & 1\\
4 &abaa\$ & Border kann nicht direkt erweitert werden$\ ^1$, aber Border von BorderTabelle[1] kann fortgesetzt werd & 1\\
5 &abaab\$ & Border kann hier erweitert werden$\ ^2$ & 2\\
6 &abaaba\$ & Border kann hier erweitert werden$\ ^2$ & 3\\
7 &abaabab\$ & Border kann nicht direkt erweitert werden$\ ^1$ aber Border von BorderTabelle[3] kann fortgesetzt werden& 2\\
8 &abaababa\$ & Border kann hier erweitert werden$\ ^2$ & 3\\
9 &abaababab\$ & Border kann nicht direkt erweitert werden$\ ^1$ aber Border von BorderTabelle[3] kann fortgesetzt werden & 2\\
10&abaabababa\$ & Border kann hier erweitert werden$\ ^2$ & 3\\
11&abaabababab\$ & Border kann nicht direkt erweitert werden$\ ^1$ aber Border von BorderTabelle[3] kann fortgesetzt werden & 2\\
12&abaababababa\$ & Border kann hier erweitert werden$\ ^2$ & 3\\
13&abaababababaa\$ & Border kann hier erweitert werden$\ ^2$ & 4\\
14&abaababababaab\$ & Border kann hier erweitert werden$\ ^2$ & 5\\
15&abaababababaabb\$ & Kein Border kann hier fortgesetzt werden & 0
\end{tabular}
\section*{Aufgabe 3}
Sei $\Sigma = \{a,b\}$, $t$ der Text mit Länge $n$, $s$ ein Suchmuster der Länge $m$ und sei $s$ dergestalt: $a^kb$. Um die geforderte
Bedingung von $2n -m$ Vergleichen zu erfüllen, muss $t$ folgendermaßen aussehen: $a^{k'}b$, wobei $k'\geq k$.\newline Beispiele $(n,m)$:\newline
\begin{itemize}
 \item Fall 1: $n=m$:\newline \texttt{t: aa...b}\newline \texttt{s: aa...b}\newline Es gibt dann $n= 2n-n = 2n-m$ Vergleiche, Bedingung erfüllt.
 \item Fall 2: $n>m$:\newline
\begin{itemize}
 \item Bsp. 2.1: $n=5$, $m=3$:\newline \texttt{...01234}\newline \texttt{t: aaaab}\newline \texttt{...||-}\newline\texttt{s: aab} \newline
 Mismatch bei $j=2$, $s$ wird auf Position $2-1=1$ verschoben, da $|\partial(aa)| = |a| = 1$, der nächste Vergleich startet bei $j=2$:\newline
 \texttt{...01234}\newline \texttt{t: aaaab}\newline \texttt{.....|-}\newline\texttt{s: .aab} \newline
  Mismatch bei $j=3$, $s$ wird auf Position $3-1=2$ verschoben, da $|\partial(aa)| = |a| = 1$, der nächste Vergleich startet bei $j=3$:\newline
  \texttt{...01234}\newline \texttt{t: aaaab}\newline \texttt{......||}\newline\texttt{s: ..aab} \newline
  Insgesamt $7 = 2*5-3 = 2n-m$ Vergleiche, Bedingung erfüllt.\newline Es fällt auf, dass bis auf die ersten $m-1$ Positionen jede Position
  2 mal einem Vergleich unterzogen wird. Dies liegt daran, dass $s$ jew. nur um eine Position verschoben werden kann.
  \item 2.2: Allgemeiner Fall:\newline
  Das oben beschriebene Verhalten gilt für alle $n>m$: Die ersten $m-1$ Positionen werden genau einmal verglichen, an allen anderen, bis auf
  die letzte, also $n-1$-te Position, finden 2 Vergleiche statt, da $s$ nur um eine Position nach rechts verschoben wird, also insgesamt
  $2n-(m-1)-1 = 2n-m$ Vergleiche. Das gilt für alle $n>m$.\newline Des weiteren tritt bis auf das letzte Mal in
  $s$ bei $j=m-1$ ein Mismatch auf, da dann $t[i+j] = a$ und $s[j]=b$, außer $i+j=n-1$. Wie weit $s$ verschoben wird, entscheidet
  $m-border[j]$. Allerdings ist das immer gleich 1, da der der längste echte Rand von $a^{m-1}$ um eins kleiner ist als $m$.\newline
  Es folgt also: solange $n>m$ ist wird $s$ jeweils nur um 1 Position nach rechts verschoben, wobei in $t$ an allen Positionen $i$ von $t[m]$ bis
  $t[n-2]$ ein Mismatch auftreten wird. Daraufhin wird $s$ um 1 weitergeschoben und an Pos. $i$ \textit{ein weiterer Vergleich} durchgeführt,
  der immer zu einem Match führt (a=a). Der nächste Vergleich bei $i+1$ wird jedoch immer zu einem Mismatch (a$\neq$b) führen usw.
\end{itemize}

\end{itemize}

\section*{Aufgabe 4}
\end{document}
