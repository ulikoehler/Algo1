\documentclass[a4paper,12pt,leqno]{scrartcl}
\usepackage{listingsutf8}
\usepackage[pdftex]{graphicx}
\usepackage[ngerman]{babel}
\usepackage{url}
\usepackage{hyperref}
\usepackage{amsfonts}
\usepackage{amssymb}
\usepackage{amsmath}
\usepackage[utf8x]{inputenc}
\usepackage[T1]{fontenc}
\usepackage{lmodern} %Latin modern = enhanced CM font
\usepackage{xspace} %Space enhancements
\usepackage[tracking=true,activate={true,nocompatibility},babel=true]{microtype} %PDFTeX typography enhancements
\usepackage{fixltx2e}
\usepackage{color}
\usepackage{array}
\usepackage{float} %Configure figure floats to be boxed
  \floatstyle{boxed}
  \restylefloat{figure}

\setlength{\parindent}{0mm}

\newcommand{\textrightarrow}{$\rightarrow$}


\lstset{
   basicstyle=\scriptsize\ttfamily,
   keywordstyle=\bfseries\ttfamily\color{blue},
   stringstyle=\color{orange}\ttfamily,
   commentstyle=\color{green}\ttfamily,
   emph={square}, 
   emphstyle=\color{blue}\texttt,
   emph={[2]root,base},
   emphstyle={[2]\color{yac}\texttt},
   showstringspaces=false,
   flexiblecolumns=false,
   tabsize=2,
   numbers=left,
   numberstyle=\tiny,
   numberblanklines=false,
   stepnumber=1,
   numbersep=10pt,
   xleftmargin=15pt,
   language=Java
 }


\definecolor{middlegray}{rgb}{0.5,0.5,0.5}
\definecolor{lightgray}{rgb}{0.8,0.8,0.8}
\definecolor{orange}{rgb}{0.8,0.3,0.3}
\definecolor{yac}{rgb}{0.6,0.6,0.1}


\begin{document}
%% DO NOT ALTER UNLESS YOU KNOW EXACTLY WHAT YOURE DOING
{\scriptsize
\noindent\begin{tabular*}{\textwidth}{@{\extracolsep{\fill}}lr}
& Uli Köhler (10580373)\\
& Tobias Harrer (XXXXXX)
\end{tabular*}
}
\\
\begin{center}
\large 0. Übung zur Vorlesung\\
\Huge Algorithmische Bioinformatik 1
\end{center}\vspace{1cm}
%% END OF DO NOT ALTER
%% Add the main content here
\section*{Aufgabe 1:}
\subsection*{(a)}
Induktionsanfang: $n = k$\\
\begin{align*}
\sum^n_{i=k} \binom{i}{k} &= \binom{n+1}{k+1}\\
\binom{k}{k} &= \binom{n+1}{k+1}\\
\text{Da n $\geq$ k:}\\
\frac{k!}{k!\cdot (k-k)!} &= \frac{(k+1)!}{(k+1)!\cdot [(k+1)-(k+1)]!}\\
\frac{k!}{k!\cdot 0!} &= \frac{(k+1)!}{(k+1)!\cdot 0!}\\
\frac{1}{1\cdot 0} &= \frac{1}{1\cdot 0}\ q.e.d
\end{align*}

\paragraph{Induktionsfortsetzung:}
Die Linke Seite der zu zeigenden Gleichung sei im Folgenden definiert als $L_k(n)$.
Analog sei die rechte Seite definiert als $R_k(n)$\\

Zu zeigen ist daher $L_k(n+1) = R_k(n+1)$, sofern $L_k(n) = R_k(n)$ gilt.

Es gilt:
\begin{align*}
L_k(n+1) &= L_k(n) + \binom{n+1}{k}\\
R_k(n+1) &= \frac{(n+2)!}{k!\cdot [(n+2)-k]!}\\
&= \frac{n!\cdot (n+1)}{k!\cdot [(n+1)-k]!\cdot [(n+2)-k]\\


\end{align*}

Da gilt $L_k(n) = R_k(n) = $

\section*{Aufgabe 2:}

\end{document}
